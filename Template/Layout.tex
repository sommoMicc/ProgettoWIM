\documentclass[11pt,a4paper]{article}

\usepackage[T1]{fontenc}
\usepackage[utf8x]{inputenc}
\usepackage[italian]{babel}
\usepackage{helvet}
\usepackage{multirow}
\renewcommand*\familydefault{\sfdefault}
\linespread{1.2}
\usepackage{fancyhdr}
\usepackage{lastpage}
\usepackage{graphicx}
\usepackage{tabularx}
\usepackage{hyperref}

\usepackage[
a4paper,
top=2.5cm,
bottom=2.5cm,
left=1.5cm,
right=1.5cm,
head=30pt,
textheight=8in,
footskip=10pt
]{geometry}

\usepackage[italian]{isodate}
\usepackage{arydshln}
\usepackage{hyperref}
\usepackage{amsfonts, amsmath, amsthm, amssymb}
\usepackage{eurosym}

% Specifies custom hyphenation points for words or words that shouldn't be hyphenated at all
\hyphenation{ionto-pho-re-tic iso-tro-pic fortran} 
\hypersetup{
	colorlinks=true,
	linkcolor=black,
	urlcolor=blue
}

%----------------------------------------------------------------------------------------
%	PAGE STYLE
% ---------------------------------------------------------------------------------------
\pagestyle{fancy}
\fancyhf{}
\setlength{\headheight}{2cm} 

% Delete the paragraph indentation
\setlength{\parindent}{0pt}

%----------------------------------------------------------------------------------------
%	UNIPD STYLE
% ---------------------------------------------------------------------------------------
\newcommand{\stileUNIPD}{
	
	% header
	\lhead{
		\textline[t]{\includegraphics[width=1cm, keepaspectratio=true]{img/UniPd.png}}
		{Progetto di Web Information Management}
		{\nomeStudente \cognomeStudente\\ (\matricolaStudente)}
	}
	
	% footer
	\rfoot{\thepage/\pageref{LastPage}} %per le prime pagine: mostra solo il numero romano
	\cfoot{}
	
}


%----------------------------------------------------------------------------------------
% 	HEADING STILE
%----------------------------------------------------------------------------------------

\newcommand\textline[4][t]{%
	\noindent\parbox[#1]{.333\textwidth}{\raisebox{-0.40\height}{#2}}%
	\parbox[#1]{.333\textwidth}{\centering #3}%
	\parbox[#1]{.333\textwidth}{\raggedleft #4}%
}

% Header delle pagine UNIPD style, commentarle se si intende usare lo stile aziendale

\stileUNIPD

% Header delle pagine stile aziendale, usaree, dopo aver chiesto il consenso all'azienda
% per l'uso del logo e dei dati personali. Commentarle se si intende usare l'UNIPD style

%\stileAziendale

% Line under the heading
\renewcommand{\headrulewidth}{0.4pt}  

%-----------------------------------------------------------------------------------------
%   FOOTER STYLE
%-----------------------------------------------------------------------------------------

%***PIÈ DI PAGINA***
%\lfoot{\includegraphics[keepaspectratio = true, width = 25px] {img/UniPd.png} \textit{\nomeStudente 
%\cognomeStudente (\matricolaStudente)}\\ \footnotesize{\emailStudente}}
%\rfoot{\thepage/\pageref{LastPage}} %per le prime pagine: mostra solo il numero romano


\renewcommand{\footrulewidth}{0.4pt}   %Linea sopra il piè di pagina

\let\oldsection\section
\renewcommand\section{\clearpage\oldsection}
