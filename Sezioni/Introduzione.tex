\section{Analisi Preliminare}
MyProtein è un brand specializzato in prodotti alimentari e abbigliamento per sportivi. Il marchio, con sede a Manchester, è attivo dal 2004 su scala globale. Per commercializzare i propri prodotti, MyProtein si affida alla vendita online, sia su Amazon che, soprattutto, sul portale myprotein.it (oggetto di questa analisi). Il brand e di conseguenza anche il sito hanno subito un importante restyling al termine del 2018.\\
\subsection{Dominio}
Il dominio scelto è myprotein.it, che è il nome del brand. L'utilizzo del \textit{top-level domain} ".it" è dettato dall'esigenza di internazionalizzazione: esiste anche il sito myprotein.com, scritto in inglese con prezzi in sterline, rivolto al mercato del Regno Unito.\\
Per quanto riguarda il nome del brand, esso è facilmente memorizzabile in quanto associa due parole di uso comune ("my" e "protein"). A mio avviso, quando si parla di integrazione alimentare sportiva, il primo prodotto che viene in mente sono le proteine in polvere ("protein" in inglese), pertanto, pur non essendo un nome particolarmente corto, è comunque molto facilmente memorizzabile per un pubblico di sportivi, che è il target di utenza al quale si rivolge.