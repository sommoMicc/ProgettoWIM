\section{Considerazioni finali e conclusioni}
Vorrei concludere questa analisi esprimendo alcune considerazioni, divise per argomento.
\paragraph{Design} Trovo molto piacevole il design del sito. Gli abbinamenti cromatici sono azzeccati, non sono presenti sezioni troppo colorate o fastidiose, l'aspetto genereale è abbastanza pulito (non c'è bloated-design) e permette all'utente di focalizzarsi sul contenuto piuttosto che sull'aspetto (non c'è il fenomeno del lorem-ipsum). Una cosa che non ho menzionato, è che il design del sito è responsivo e si adatta abbastanza bene a tutte le tipologie di schermi, dallo smartphone fino al pc.
\paragraph{Pubblicità} assente.
\paragraph{Testo} Il testo presente nelle pagine è generalmente composto di frasi molto semplici, corte e in generale non è molto complicato. Inoltre, è assente lo scroll orizzontale.
\paragraph{Login/registrazione} Tutto il sito è accessibile senza la necessità di dover avere un account, che è necessario invece per completare un ordine. Trovo questa scelta molto sensata, in quanto sarebbe quasi masochismo precludere l'accesso a dei visitatori solo perché non hanno un account.\\In ogni caso, il modulo di registrazione è abbastanza breve: email, conferma email, password, conferma password, nome, per un totale di 5 campi di testo. Segnalo inoltre che, in fase di registrazione, è possibile utilizzare il proprio account Facebook, velocizzando così la procedura. 
\paragraph{Popup e back-button} Nel sito non sono presenti pop-up, e il comportamento del pulsante indietro è quasi sempre quello atteso. Quando viene aperta una finestra modale, come quella visualizzata in figura \ref{fig:figura10}, un utente poco esperto potrebbe essere tentato a premere sul tasto indietro del browser per chiuderla. Come già detto in precedenza, io la eliminerei.\\
\\ E per finire:
\paragraph{Conclusioni} Nonostante alcune defaillance, ritengo MyProtein un sito ben realizzato, in grado di regalare all'utente un'esperienza mediamente più che positiva, ma che ha ancora del potenziale di miglioramento.\\
\\ \\{\huge Voto: 7.5}

\section{Lista delle figure}
\begin{center}
    \begin{tabular}{ | c | c | c | c | }
        \hline
        \textbf{Figura} & \textbf{File} & \textbf{URL} \\ 
        \hline
        Figura \ref{fig:figura1} & figura1.jpg & https://www.myprotein.it/ \\
        \hline
        Figura \ref{fig:figura2} & figura2.jpg & https://www.myprotein.it/ \\
        \hline
        Figura \ref{fig:figura3} & figura3.jpg & https://www.myprotein.it/ \\
        \hline
        Figura \ref{fig:figura4} & figura4.png & https://www.myprotein.it/ \\
        \hline
        Figura \ref{fig:figura5} & figura5.png & https://www.myprotein.it/ \\
        \hline
        Figura \ref{fig:figura6} & figura6.png & https://www.myprotein.it/ \\
        \hline
        Figura \ref{fig:figura7} & figura7.png & https://www.myprotein.it/elysium.search?search=proteine \\
        \hline
        Figura \ref{fig:figura8} & figura8.png & https://www.myprotein.it/elysium.search?search=colomba+pasquale \\
        \hline
        Figura \ref{fig:figura9} & figura9.png & https://www.myprotein.it/elysium.search?search=colomba+pasquale \\
        \hline
        Figura \ref{fig:figura10} & figura10.png & https://www.myprotein.it/nutrizione-sportiva/impact-whey-protein-elite\\
        \hline
        Figura \ref{fig:figura11} & figura11.png & https://www.myprotein.it/my.basket \\
        \hline
        Figura \ref{fig:figura12} & figura12.png & https://www.myprotein.it/dummy \\
        \hline
        Figura \ref{fig:figura13} & figura13.png & https://www.myprotein.it/ \\
        \hline
    \end{tabular}
\end{center}
